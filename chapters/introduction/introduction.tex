\chapter{Introduction}\label{chapter:introduction}

Publish-subscribe is a messaging pattern where the senders of the messages are called \textit{publishers}, and the receivers of the messages are called \textit{subscribers}. The messages from publishers are characterized into classes. The subscribers receive messages from one or more classes to which they express their interest. Publish-subscribe pattern has a widespread adoption in distributed systems due to a high degree of decoupling. This is because the publishers and subscribers have no knowledge of each other. The co-ordination is done by an entity called the broker. Both the publishers and subscribers only communicate with the broker. The publisher publishes data without any information regarding the identity, location or the number of subscribers. Similarly, the subscribers receive the data without any information regarding the publishers. The other main advantage of this pattern is scalability. Publish-subscribe pattern provides the room for better scalability compared to traditional client-server due to message caching, parallel operation, network-based routing, etc. The earliest mention of publish-subscribe systems was the \textit{news} subsystem of the Isis Toolkit \parencite{Birman:1987:EVS:37499.37515}.

One of the popular adoptions of this paradigm is the Topic-based publish-subscribe pattern. In this approach there exists a named logical channel called \textit{topics}. All the subscribers will receive the same content from a topic to which the publishers publish the content.

To leverage the topic-based publish-subscribe pattern further in scenarios where bandwidth savings is important a Shared Dictionary Compression has been introduced recently \parencite{Doblander:2016:SDC}. This fairly new technique in the realm of publish-subscribe pattern aims at reducing bandwidth and in turn reducing costs. It allows the use of collaborative applications even in areas, where bandwidth is sparse.