\section{MQTT}

MQTT is a lightweight messaging protocol \parencite{locke2010mq} and an ISO standard \parencite{mqtt_iso}. It is a publish-subscribe based protocol for use on top TCP/IP. For non-TCP/IP networks used by embedded devices there exist a variant of MQTT called \textit{MQTT-SN} typically used in Wireless Sensor Networks (WSN) \parencite{4554519}. It is best suited for connections where the network bandwidth is limited. It was created by Dr. Andy Stanford-Clark and Arlen Nipper in the year 1999 \parencite{mqtt_faq}. The latest version of the protocol is V3.1. Apart from MQTT, there are other messaging protocols \parencite{messaging_protocols} such as STOMP \parencite{stomp} and AMQP \parencite{vinoski2006advanced}.

\subsection{MQTT methods}

MQTT defines methods to perform actions on selected resources. The representation of the resources depends on the implementation of the server. The resource could be a file or any executable on the server.

Below are the methods defined by MQTT:

\begin{itemize}
    \item\textbf{Connect:}
        Waits for the MQTT client to establish a connection with the server.

    \item\textbf{Disconnect:}
        Waits till the MQTT client completes its work and for the session to disconnect.

    \item\textbf{Subscribe:}
        Waits for the subscription to complete.

    \item\textbf{UnSubscribe:}
        Asks the server to unsubscribe a client from one or multiple topics.

    \item\textbf{Publish:}
        Passes the request to the MQTT client.

\end{itemize}

\subsection{Security}

The latest version of the MQTT protocol supports authentication by passing username and password. It also supports encryption over the network via SSL. However, as SSL is not a very light weight it adds a considerable amount of overhead. One way to avoid this is to encrypt the data that is transmitted but this is not native to the protocol.

